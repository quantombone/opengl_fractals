\documentclass[12pt,letterpaper]{article}

% tomasz malisiewicz

\usepackage{times}
\usepackage{epsfig}
\usepackage{graphicx}
\usepackage{amsmath}
\usepackage{amssymb}

% Include other packages here, before hyperref.

% If you comment hyperref and then uncomment it, you should delete
% egpaper.aux before re-running latex.  (Or just hit 'q' on the first latex
% run, let it finish, and you should be clear).
\usepackage[pagebackref=true,breaklinks=true,letterpaper=true,colorlinks,bookmarks=false]{hyperref}

\DeclareMathOperator*{\mycup}{\cup}
\DeclareMathOperator{\argmax}{argmax}
\DeclareMathOperator{\argmin}{argmin}
\DeclareMathOperator{\boldbeta}{\boldsymbol\beta}
\DeclareMathOperator{\dfun}{\boldsymbol w}
\begin{document}

\title{Newton's Method Fractals in Motion}

\author{Tomasz Malisiewicz\\
Robotics Institute \\
Carnegie Mellon University\\
Pittsburgh PA, 15232\\
{\tt\small tomasz@cmu.edu}
}

\maketitle
\abstract{Fractals generated via Newton's Method...
}

\section{Newton's Method}
Newton's method is an iterative update rule for finding the roots of
an equation $f(x)=0$.
\begin{equation}
x_{i+1} = x_{i} - \frac{f(x_i)}{f'(x_i)}
\end{equation}

For an arbitrary equation where computing the derivative $f'(x)$ is
difficult one can use the finite difference approximation
\begin{equation}
f'(x) = \frac{f(x+\epsilon)-f(x)}{\epsilon}
\end{equation}

Polynomials can be expressed two ways.  Directly encoding the zeros we
write:

\begin{equation}
f(x) = \prod_{i=1}^K(x-z_i)
\end{equation}

We can also use the more standard form

\begin{equation}
f(x) = \sum_{i=0}^K a_i x^i
\end{equation}

\end{document}